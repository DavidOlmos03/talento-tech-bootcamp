\documentclass{article}
\usepackage{listings}
\usepackage{xcolor}

% Definición de colores para código
\definecolor{codegreen}{rgb}{0,0.6,0}
\definecolor{codegray}{rgb}{0.5,0.5,0.5}
\definecolor{codepurple}{rgb}{0.58,0,0.82}
\definecolor{backcolour}{rgb}{0.95,0.95,0.92}

% Configuración de lstlisting para Python y SQL
\lstdefinestyle{mystyle}{
    backgroundcolor=\color{backcolour},   
    commentstyle=\color{codegreen},
    keywordstyle=\color{blue},
    numberstyle=\tiny\color{codegray},
    stringstyle=\color{codepurple},
    basicstyle=\ttfamily\footnotesize,
    breakatwhitespace=false,         
    breaklines=true,                 
    captionpos=b,                    
    keepspaces=true,                 
    numbers=left,                    
    numbersep=5pt,                  
    showspaces=false,                
    showstringspaces=false,
    showtabs=false,                  
    tabsize=2
}

\lstset{style=mystyle}

\title{Conexión SQL Server y Operaciones con Python}
\author{Juan David Ruiz Olmos}
\date{\today}

\begin{document}

\maketitle

\section{Paso 1: Crear un Servidor Local en SQL Server}
\begin{lstlisting}[language=SQL, caption=Crea un servidor local en SSMS]

\end{lstlisting}

\section{Paso 2: Crear un Login de SQL Authentication}
\begin{lstlisting}[language=SQL, caption=Crear un login SQL Authentication]
USE master;
GO
CREATE LOGIN MyUserLogin 
WITH PASSWORD = 'SecurePass123';
GO

CREATE USER MyUser FOR LOGIN MyUserLogin;
GO
ALTER SERVER ROLE sysadmin ADD MEMBER MyUser;
GO
\end{lstlisting}

\section{Paso 3: Crear una Base de Datos en SSMS}
\begin{enumerate}
    \item Haz clic derecho en \textbf{Databases} y selecciona \textbf{New Database}.
    \item Asigna un nombre, por ejemplo, \textbf{MyDatabase}.
    \item Haz clic en \textbf{OK} para crear la base de datos.
\end{enumerate}

\section{Paso 4: Conectar a la Base de Datos usando Python}
Asegúrate de tener instalado el paquete \texttt{pyodbc}.
\begin{lstlisting}[language=Python, caption=Código para conectarse a SQL Server]
import pyodbc


conn = pyodbc.connect(
    'DRIVER={ODBC Driver 17 for SQL Server};'
    'SERVER=localhost;'
    'DATABASE=MyDatabase;'
    'UID=MyUserLogin;'
    'PWD=SecurePass123;'
)
cursor = conn.cursor()
\end{lstlisting}

\section{Paso 5: Crear una Tabla con Mínimo 10 Registros}
\begin{lstlisting}[language=SQL, caption=Crear una tabla e insertar registros]
CREATE TABLE Employees (
    ID INT PRIMARY KEY,
    Name VARCHAR(50),
    Position VARCHAR(50)
);

INSERT INTO Employees (ID, Name, Position) VALUES
(1, 'John Doe', 'Manager'),
(2, 'Jane Smith', 'Developer'),
(3, 'Samuel Johnson', 'Analyst'),
(4, 'Michael Brown', 'Designer'),
(5, 'Linda White', 'Sales'),
(6, 'Chris Green', 'HR'),
(7, 'Sarah Black', 'Marketing'),
(8, 'Paul Blue', 'Support'),
(9, 'Laura Red', 'Finance'),
(10, 'Alice Grey', 'Consultant');
\end{lstlisting}

\section{Paso 6: Crear un Catálogo para la Tabla}
\begin{lstlisting}[language=SQL, caption=Crear un catálogo]
CREATE FULLTEXT CATALOG EmployeeCatalog AS DEFAULT;
GO
CREATE FULLTEXT INDEX ON Employees(Name)
KEY INDEX PK_Employees ON EmployeeCatalog;
GO
\end{lstlisting}

\section{Paso 7: Crear un Índice para el Catálogo}
\begin{lstlisting}[language=SQL, caption=Crear un índice]
CREATE NONCLUSTERED INDEX idx_position 
ON Employees(Position);
\end{lstlisting}

\section{Paso 8: Crear un Stored Procedure Dinámico}
\begin{lstlisting}[language=SQL, caption=Stored Procedure Dinámico]
CREATE PROCEDURE sp_GetEmployeesByPosition 
    @Position VARCHAR(50)
AS
BEGIN
    SET NOCOUNT ON;
    DECLARE @query NVARCHAR(MAX);
    SET @query = 'SELECT * FROM Employees WHERE Position = ''' + @Position + '''';
    EXEC sp_executesql @query;
END;
GO

-- Ejemplo de uso:
EXEC sp_GetEmployeesByPosition 'Developer';
\end{lstlisting}

\section{Paso 9: Crear y Utilizar Cursores}
\begin{lstlisting}[language=SQL, caption=Uso de cursores para consultas]
DECLARE EmployeeCursor CURSOR FOR
SELECT Name, Position FROM Employees;

DECLARE @Name VARCHAR(50), @Position VARCHAR(50);

OPEN EmployeeCursor;
FETCH NEXT FROM EmployeeCursor INTO @Name, @Position;

WHILE @@FETCH_STATUS = 0
BEGIN
    PRINT 'Nombre: ' + @Name + ', Cargo: ' + @Position;
    FETCH NEXT FROM EmployeeCursor INTO @Name, @Position;
END;

CLOSE EmployeeCursor;
DEALLOCATE EmployeeCursor;
\end{lstlisting}

\section{Conclusión}
En este documento, se ha demostrado cómo crear un servidor local en SQL Server, configurar un login de autenticación, crear una base de datos, y realizar diversas operaciones avanzadas como índices, procedimientos almacenados, y cursores, usando tanto SQL Server como Python.

\end{document}
